\documentclass[a4paper]{article}
\usepackage{amsmath, amssymb, amsthm}
\usepackage[text={16.5cm,23cm}]{geometry}
\usepackage{color}
\usepackage[UTF8, heading = true, fontset = windows]{ctex}
\usepackage[xetex,pdfstartview={FitH}]{hyperref}
\usepackage[utopia]{mathdesign}
\usepackage{enumerate}

\ctexset{today=small}
\ctexset{section={
name={\heiti 第,章},
number=\chinese{section},
}
}

\newcommand{\red}[1]{\textcolor{red}{#1}}
\newcommand{\e}{\mathrm{e}}
\newcommand{\disp}{\displaystyle}

\newtheoremstyle{mystyle}% name
{3pt}% Space above
{3pt}% Space below
{\kaishu}% Body font
{}% Indent amount
{\heiti}% Theorem head font
{}% Punctuation after theorem head
{.5em}% Space after theorem head
{}% Theorem head spec (can be left empty, meaning `normal')
\theoremstyle{mystyle}
\newtheorem{prob}{问题}[section]

% \renewcommand{
\begin{document}
	\title{\huge \heiti 概率论与数理统计(第四版)}
%	\author{\bf 崔峰}
	\maketitle
	\setcounter{section}{1}
	\section{\heiti 随机变量及其分布}
	\begin{prob}
		\begin{tabular}{c|ccc}
		$X$ & 20 & 5 & 0\\
		\hline
		$p_k$ & 0.0002 & 0.0010 & 0.9989
		\end{tabular}
	\end{prob}
	\setcounter{prob}{5}
	\begin{prob}
		以$X$记``同一时刻被使用的设备台数'', 则
		\[
			P(X=k)={5\choose k}(0.1)^k(0.9)^{5-k}
		\]
		\begin{enumerate}[{(}1{)}]
			\item
			$\displaystyle P(X=2)={5\choose 2}(0.1)^2(0.9)^{3}=0.0729$
			\item
			$\displaystyle 1-\sum_{k=0}^2P(X=k)=0.00856$
			\item
			$\displaystyle 1-\sum_{k=4}^5 P(X=k)=0.99954$
			\item
			$\displaystyle 1-P(X=0)=0.40951$
		\end{enumerate}
	\end{prob}
	\setcounter{prob}{10}
	\begin{prob}
		以$X$记``此地区每年撰写此类文章的篇数'', 则$X\sim \pi(6)$, 故明年没有此类文章的概率为
		\[
			P(X=0)=\frac{\lambda^0\e^{-\lambda}}{0!}\Bigg|_{\lambda=6}=0.00248.
		\]
	\end{prob}
	\setcounter{prob}{15}
	\begin{prob}
		出事故的车辆数$X$服从二项分布, 但$n$很大且$p$很小时, 可近似认为其服从泊松分布, 故令$\lambda =np=1000\times 0.0001=0.1$, 则有$X\sim\pi(0.1)$, 从而
		\[
			P(X\geqslant 2)=1-P(X=0)-P(X=1)=1-\e^{-0.1}-0.1\cdot\e^{-0.1}=0.00468.
		\]
	\end{prob}
	\setcounter{prob}{20}
	\begin{prob}
		\begin{enumerate}[{(}1{)}]
			\item
			分布函数
			\[
				F(x)=
				\begin{cases}
				0,&x<1;\\
				\disp 2x-4+\frac{2}{x}, & 1\leqslant x< 2;\\
				1, &x\geqslant2.
				\end{cases}
			\]
			\item
			分布函数
			\[
				F(x)=
				\begin{cases}
				0, & x<0;\\
				\disp\frac{x^2}{2}, & 0\leqslant x <1;\\
				\disp -\frac{x^2}{2}+2x-1, &1\leqslant x<2;\\
				1,& x\geqslant 2.
				\end{cases}
			\]
		\end{enumerate}
	\end{prob}
	\setcounter{prob}{25}
	\begin{prob}
		考虑到$X\sim N(3,2^2)$, 则有
		\begin{enumerate}[{(}1{)}]
			\item
			\begin{gather*}
				P(2<X\leqslant 5)=P(X\leqslant 5)-P(X\leqslant 2)=\varPhi(1)-\varPhi\left(-\frac{1}{2}\right)=\varPhi(1)+\varPhi\left(\frac{1}{2}\right)-1=0.53281,\\
				P(-4<X\leqslant 10)=\varPhi\left(\frac{7}{2}\right)-\varPhi\left(-\frac{7}{2}\right)=2\varPhi\left(\frac{7}{2}\right)-1=0.99953,\\
				P(|X|>2)=P(X<-2)+1-P(X\leqslant 2)=1+\varPhi\left(-\frac{5}{2}\right)-\varPhi\left(-\frac{1}{2}\right)=1+\varPhi\left(\frac{1}{2}\right)-\varPhi\left(\frac{5}{2}\right)=0.69767\\
				P(X>3)=1-P(X\leqslant 3)=1-\varPhi(0)=0.5
			\end{gather*}
			\item
			即求$1-P(X\leqslant c)=P(X\leqslant c)$, 即
			\[
				P(X\leqslant c)=\varPhi\left(\frac{c-3}{2}\right)=\frac{1}{2},
			\]
			所以有$c=3$.
			\item
			由$P(X>d)\geqslant 0.9$, 可知$1-P(X\leqslant d)\geqslant 0.9$, 即$P(X\leqslant d)\leqslant 0.1$, 而$\varPhi(-1.28155)=0.1$, 所以
			\[
				d = 2\times(-1.28155) +3=0.43690.
			\]
		\end{enumerate}
	\end{prob}
\end{document}
