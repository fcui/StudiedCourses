\documentclass[a4paper]{article}
\usepackage{amsmath, amssymb, amsfonts,amsthm}
\usepackage[text={16.5cm,23cm}]{geometry}
\usepackage{color}
\usepackage[UTF8, heading = true, fontset = windowsnew]{ctex}
\usepackage[xetex,pdfstartview={FitH}]{hyperref}
\usepackage[utopia]{mathdesign}
\usepackage{enumerate}

\ctexset{today=small}
\ctexset{section={
name={第,章},
number=\chinese{section},
}
}

\newcommand{\red}[1]{\textcolor{red}{#1}}

\newtheoremstyle{mystyle}% name
{3pt}% Space above
{3pt}% Space below
{\kaishu}% Body font
{}% Indent amount
{\bfseries}% Theorem head font
{}% Punctuation after theorem head
{.5em}% Space after theorem head
{}% Theorem head spec (can be left empty, meaning `normal')
\theoremstyle{mystyle}
\newtheorem{prob}{问题}[section]

% \renewcommand{
\begin{document}
	\title{\huge \bf 概率论与数理统计(第四版)}
	\author{\bf 崔峰}
	\maketitle
	\setcounter{section}{1}
	\section{随机变量及其分布}
	\begin{prob}
		\begin{tabular}{c|ccc}
		$X$ & 20 & 5 & 0\\
		\hline
		$p_k$ & 0.0002 & 0.0010 & 0.9989
		\end{tabular}
	\end{prob}
	\setcounter{prob}{5}
	\begin{prob}
		以$X$记``同一时刻被使用的设备台数'', 则
		\[
			P(X=k)={5\choose k}(0.1)^k(0.9)^{5-k}
		\]
		\begin{enumerate}[{(}1{)}]
			\item
			$\displaystyle P(X=2)={5\choose 2}(0.1)^2(0.9)^{3}=0.0729$
			\item
			$\displaystyle 1-\sum_{k=0}^2P(X=k)=0.00856$
			\item
			$\displaystyle 1-\sum_{k=4}^5 P(X=k)=0.99954$
			\item
			$\displaystyle 1-P(X=0)=0.40951$
		\end{enumerate}
	\end{prob}
\end{document}
