\section{随机变量的数字特征}
\subsection{数学期望}
\begin{prob}
	\begin{enumerate}
	\item 
	以$X$表示取到的单词所包含的字母个数, 则其分布律为
	\begin{center}
		\begin{tabular}{c|cccc}
			\hline
			$X$&2&3&4&9\\
			\hline
			$p_k$&$\frac{1}{8}$&$\frac{5}{8}$&$\frac{1}{8}$&$\frac{1}{8}$\\
			\hline
		\end{tabular}
	\end{center}
	则期望$\disp E(X)=\frac{15}{4}=3.75$.
	\item
	以$Y$表示取到的字母所在单词所包含的字母数, 则其分布律为
	\begin{center}
		\begin{tabular}{c|cccc}
			\hline
			$Y$&2&3&4&9\\
			\hline
			$p_k$&$\frac{1}{15}$&$\frac{1}{2}$&$\frac{2}{15}$&$\frac{3}{10}$\\
			\hline
		\end{tabular}
	\end{center}
	则期望$\disp E(Y)=\frac{73}{15}=4.87$.
	\item
	得到分布律为
	\begin{center}
		\begin{tabular}{c|ccccccccccc}
			\hline
			$X$&1&2&3&4&5&7&8&9&10&11&12\\
			\hline
			$p_k$&$\frac{1}{6}$&$\frac{1}{6}$&$\frac{1}{6}$&$\frac{1}{6}$&$\frac{1}{6}$
			&$\frac{1}{36}$&$\frac{1}{36}$&$\frac{1}{36}$&$\frac{1}{36}$&$\frac{1}{36}$&$\frac{1}{36}$\\
			\hline
		\end{tabular}
	\end{center}
	则期望$\disp E(Y)=\frac{49}{12}=4.08$.
	\end{enumerate}
\end{prob}
\setcounter{prob}{5}
\begin{prob}
	\begin{enumerate}
	\item 
	容易计算得到
	\begin{gather*}
		E(X)=(-2)\times 0.4+0\times 0.3+2\times 0.3=-0.2,\\
		E(X^2)=(-2)^2\times 0.4+0^2\times 0.3+2^2\times 0.3=2.8,\\
		E(3X^2+5)=17\times 0.4+5\times 0.3+17\times 0.3=13.4,\\
		\red{E(3X^2+5)=3E(X^2)+5=13.4}.
	\end{gather*}
	\item 
	因为$X\sim\pi(\lambda)$, 即
	\[
		P\{X=k\}=\frac{\lambda^k\e^{-\lambda}}{k!},\quad k=0,1,2,\ldots.
	\]
	于是有
	\[
		E[1/(X+1)]=\sum_{k=0}^\infty \frac{1}{k+1}\cdot\frac{\lambda^k\e^{-\lambda}}{k!}
		=\frac{\e^{-\lambda}}{\lambda}\sum_{k=0}^\infty \frac{\lambda^{k+1}}{(k+1)!}
		=\frac{\e^{-\lambda}}{\lambda}\left(\e^\lambda-1\right)=\frac{1-\e^{-\lambda}}{\lambda}.
	\]
	\end{enumerate}
\end{prob}
\setcounter{prob}{10}
\begin{prob}
	因为使用寿命小于1年的概率为
	\[
		P\{X\leqslant 1\}=\int_0^1\frac{1}{4}\e^{-x/4}\dd x=1-\e^{-1/4}, 
	\]
	则$P\{X>1\}=\e^{-1/4}$. 于是赢利的数学期望为
	\[
		100\times\e^{-1/4}-\red{200}\times(1-\e^{-1/4})=33.6402.
	\]
\end{prob}
\setcounter{prob}{15}
\begin{prob}
	\begin{enumerate}
	\item 
	写出分布律
	\begin{center}
		\begin{tabular}{c|cccc}
			\hline
			$X$ & 1 & 2 & $\cdots$ &$n$\\
			\hline
			$p_k$ & $\frac{1}{n}$ & $\frac{n-1}{n}\cdot\frac{1}{n-1}$ & $\cdots$ &$\frac{1}{n}$\\
			\hline
		\end{tabular}
	\end{center}
	所以期望是
	\[
		E(X)=\sum_{i=1}^{n}i\cdot \frac{1}{n}=\frac{n+1}{2}.
	\]
	\red{
	\item
	如果不写出$X$分布律, 这时可假设有随机变量
		\begin{align*}
			X_1&=1,\\
			X_k&=
			\begin{cases}
				1, &\textrm{前}k-1\textrm{次均未找到合适的钥匙};\\
				0, &\textrm{前}k-1\textrm{次中有一次试开成功}.
			\end{cases} 
			\qquad k=2,3,\ldots,n.
		\end{align*}
		那么$E(X_1)=1$, 同时对于$k=2,3,\ldots, n$可以求得
		\begin{gather*}
			P\{X_k=1\}=\frac{n-1}{n}\cdot\frac{n-2}{n-1}\cdots\frac{n-(k-1)}{n-k}=\frac{n-k+1}{n},\\
			E(X_k)=1\cdot P\{X_k=1\}+0\cdot P\{X_k=0\}=\frac{n-k+1}{n}.
		\end{gather*}
		于是最后我们有
		\[
			E(X)=1+\sum_{k=2}^nE(X_k)=1+\sum_{k=2}^n \frac{n-k+1}{n}=\frac{n+1}{2}.
		\]
	}
	\end{enumerate}
\end{prob}
\subsection{方差}
\setcounter{prob}{20}
\begin{prob}
	求得期望
	\[
		E(X)=\sum_{k=1}^{\infty}k\cdot p(1-p)^{k-1}=p\sum_{k=1}^{\infty}k(1-p)^{k-1}=\frac{1}{p},
	\]
	而方差
	\begin{gather*}
		E(X^2)=\sum_{k=1}^{\infty}k^2\cdot p(1-p)^{k-1}=p\sum_{k=1}^{\infty}k^2\cdot (1-p)^{k-1}
		=p\cdot\frac{2-p}{p^3}=\frac{2-p}{p^2},\\
		D(X)=E(X^2)-[E(X)]^2=\frac{1-p}{p^2}.
	\end{gather*}
\end{prob}
\setcounter{prob}{25}
\begin{prob}
	\begin{enumerate}
	\item 
	\end{enumerate}
\end{prob}