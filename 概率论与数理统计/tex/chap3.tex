\section{多维随机变量及其分布}
\subsection{二维随机变量}
\begin{prob}
	\begin{enumerate}
	\item 	
	\renewcommand{\arraystretch}{1.7}
	\begin{tabular}{c|cc}
		\hline
		\diagbox[width=2cm, height=1.2cm]{$Y$}{$X$}&0&1\\
		\hline
		0& $\disp\frac{25}{36}$& $\disp\frac{5}{36}$\\
		1& $\disp\frac{5}{36}$ & $\disp\frac{1}{36}$\\
		\hline
	\end{tabular}
	\item 	
		\begin{tabular}{c|cc}
			\hline
			\diagbox[width=2cm, height=1.2cm]{$Y$}{$X$}&0&1\\
			\hline
			0& $\disp\frac{45}{66}$& $\disp\frac{5}{33}$\\
			1& $\disp\frac{5}{33}$& $\disp\frac{1}{66}$\\
			\hline
		\end{tabular}
	\end{enumerate}
\end{prob}
\subsection{边缘分布}
\setcounter{prob}{5}
\begin{prob}\red{
	参见课本第2页中样本空间$S_2$的8个样本点, $X$和$Y$取值情况如下:
	\begin{center}
		\begin{tabular}{c|cccccccc}
			\hline
			样本点& $HHH$&$HHT$&$HTH$&$THH$&$HTT$&$THT$&$TTH$&$TTT$\\
			\hline
			$X$&2&2&1&1&1&1&0&0\\
			\hline
			$Y$&3&2&2&2&1&1&1&0\\
			\hline
		\end{tabular}
	\end{center}
		基于此可得$X$和$Y$的联合分布律和边缘分布律为:
		\begin{center}
			\renewcommand{\arraystretch}{1.7}
			\begin{tabular}{c|ccc|c}
				\hline
				\diagbox[width=2cm, height=1.2cm]{$Y$}{$X$}&0&1&2&$P\{Y=j\}$\\
				\hline
				0&$\disp\frac{1}{8}$&0&0&$\disp\frac{1}{8}$\\
				1&$\disp\frac{1}{8}$&$\disp\frac{1}{4}$&0&$\disp\frac{3}{8}$\\
				2&0&$\disp\frac{1}{4}$&$\disp\frac{1}{8}$&$\disp\frac{3}{8}$\\
				3&0&0&$\disp\frac{1}{8}$&$\disp\frac{1}{8}$\\
				\hline
				$P\{X=i\}$&$\disp\frac{1}{4}$&$\disp\frac{1}{2}$&$\disp\frac{1}{4}$&1\\
				\hline				
			\end{tabular}
		\end{center}}
\end{prob}
\subsection{条件分布}
\setcounter{prob}{10}
\begin{prob}
	\begin{enumerate}
		\item
		边缘分布律:
		\begin{align*}
			p_{n\cdot}=P\{X=n\}&=\frac{\e^{-14}}{n!}\sum_{m=0}^n n!\frac{7.14^m 6.86^{n-m}}{m!(n-m)!}
			=\frac{\e^{-14}}{n!}\sum_{m=0}^n{n\choose m}7.14^m 6.86^{n-m}\\
			&=\frac{\e^{-14}}{n!}(7.14+6.86)^n = \frac{14^n\e^{-14}}{n!}, \qquad n=0,1,2,\ldots.
		\end{align*}
		\red{说明$X$服从参数为$14$的泊松分布, 即$X\sim\pi(14)$}.
		\begin{align*}
			p_{\cdot m}=P\{Y=m\}&=\frac{7.14^m \e^{-14}}{m!}\sum_{n=m}^{+\infty} \frac{6.86^{n-m}}{(n-m)!}
			=\frac{7.14^m \e^{-14}}{m!}\sum_{n=0}^{+\infty}\frac{6.86^n}{n!}\\
			&=\frac{7.14^m \e^{-14}\e^{6.86}}{m!}=\frac{7.14^m \e^{-7.14}}{m!}, \qquad m=0,1,2,\ldots
		\end{align*}
		\red{说明$Y$服从参数为$7.14$的泊松分布, 即$X\sim\pi(7.14)$}.
		\item
		条件分布律:
		\begin{gather*}
			P\{Y=m\mid X=n\}={n\choose m}\frac{7.14^m 6.86^{n-m}}{14^n}={n\choose m}0.51^m 0.49^{n-m},\quad m=0,1,2,\ldots,n.\\
			P\{X=n\mid Y=m\}=\frac{6.86^{n-m}\e^{-6.86}}{(n-m)!},\quad (\red{n=m,m+1,\ldots})
		\end{gather*}
		\item
		$X=20$时, $Y$的条件分布律为$\disp {20\choose m}0.51^m 0.49^{20-m}$, $m=0,1,\ldots,20$.
	\end{enumerate}
\end{prob}
\subsection{相互独立的随机变量}
\setcounter{prob}{16}
\begin{prob}
	\begin{enumerate}
	\item 	
	\red{求出边缘分布律
	\begin{gather*}
		F_X(x)=F(x,+\infty)=
		\begin{cases}
			1-\e^{-\alpha x}, & x\geqslant 0;\\
			0, & \textrm{其他}.
		\end{cases}\\
		F_Y(y)=F(+\infty, y)=
		\begin{cases}
		y,& 0\leqslant y\leqslant 1;\\
		1, & y>1;\\
		0, &\textrm{其他}.
		\end{cases}
	\end{gather*}
	由此即可知 $F(x,y)=F_X(x)F_Y(y)$, 从而$X, ~Y$相互独立.}
	\item
	考虑到对于$0<p<1$, 有
	\begin{gather*}
		P\{X=x\}=p^2(1-p)^{x-1}\red{\sum_{y=1}^\infty (1-p)^{y-1}}=p(1-p)^{x-1},\quad x=1,2,\ldots,\\
		P\{Y=y\}=p^2(1-p)^{y-1}\red{\sum_{x=1}^\infty (1-p)^{x-1}}=p(1-p)^{y-1},\quad y=1,2,\ldots.
	\end{gather*}
	从而$P\{X=x,Y=y\}=P\{X=x\}P\{Y=y\}$, 故$X,~Y$相互独立.
	\end{enumerate}
\end{prob}
\subsection{两个随机变量的函数的分布}
\setcounter{prob}{20}
\begin{prob}
	\begin{enumerate}
	\item 
	由公式知
	\[
		f_Z(z)=\int_{-\infty}^{+\infty} f(x,z-x)\dd x, \quad 0<x<1,\quad x<z<x+1.
	\]
	\red{于是通过示意图可知
	\[
		f_Z(z)=
		\begin{cases}
			\disp\int_0^z z\dd x=z^2,&0<z<1;\\
			\disp\int_{z-1}^1 z\dd x=z(2-z), & 1\leqslant z<2;\\
			0,& \textrm{其他}.
		\end{cases}
	\]}
	\item
	由公式知
	\[
		f_Z(z)=\int_{-\infty}^{+\infty}\frac{1}{|x|}f\left(x,\frac{z}{x}\right)\dd x,
		\quad 0<x<1,\quad 0<z<x.
	\]
	画出示意图后可知
	\[
		f_Z(z)=
		\begin{cases}
			\disp \int_z^1 \frac{1}{x}\left(x+\frac{z}{x}\right)\dd x=2(1-z), & 0< z<1;\\
			0, &\textrm{其他}.
		\end{cases}
	\]
	\end{enumerate}
\end{prob}
\setcounter{prob}{25}
\begin{prob}
	由公式可知
	\[
		f_Z(z)=
		\begin{cases}
			\disp\int_{-\infty}^{+\infty}|x|f(x,xz)\dd x=\int_0^{+\infty}x\e^{-x}\e^{-xz}\dd x=\frac{1}{1+z^2},&z>0;\\
			0, &\textrm{其他}.
		\end{cases}
	\]
\end{prob}
\setcounter{prob}{30}
\begin{prob}
	\begin{enumerate}
	\item \red{
	由瑞利分布的概率密度函数
	\[
		f_X(x)=
		\begin{cases}
			\disp\frac{x}{4}\e^{-x^2/8},& x\geqslant 0;\\
			0, &\textrm{其他}.
		\end{cases}
	\]
	可知其分布函数为
	\[
		F_X(x)=
		\begin{cases}
			\disp 1-\e^{-x^2/8},& x\geqslant 0;\\
			0, &\textrm{其他}.
		\end{cases}
	\]}
	\red{由于$X_i(i=1,2,3,4,5)$是独立同分布的随机变量}, 所以有
	\[
		F_{\max}(z)=
		\begin{cases}
			\disp \left(1-\e^{-z^2/8}\right)^5,& z\geqslant 0;\\
			0, &\textrm{其他}.
		\end{cases}
	\]
	\item
	容易知道
	\[
		P\{Z>4\}=1-F_{\max}(z=4)=0.51668.
	\]
	\end{enumerate}
\end{prob}
\setcounter{prob}{35}
\begin{prob}
	\begin{enumerate}
	\item
	容易计算
	\begin{gather*}
		P\{X=2\mid Y=2\}=\frac{P\{X=2, Y=2\}}{P\{Y=2\}}=\frac{0.05}{0.25}=\frac{1}{5},\\
		P\{Y=3\mid X=0\}=\frac{P\{Y=3, X=0\}}{P\{X=0\}}=\frac{0.01}{0.03}=\frac{1}{3}.
	\end{gather*} 
	\item
	$V=\max\{X,Y\}$的分布律为
	\begin{center}
		\begin{tabular}{c|cccccc}
			\hline
			$V$&0&1&2&3&4&5\\
			\hline
			$p_k$&0&0.04&0.16& 0.28&0.24&0.28\\
			\hline
		\end{tabular}
	\end{center}
	以其中一个为例
	\begin{multline*}
		P\{V=3\}=P\{\max\{X,Y\}=3\}\\
		=P\{X=3,Y<3\}+P\{X<3,Y=3\}+P\{X=3,Y=3\}=0.15+0.07+0.06=0.28.
	\end{multline*}
	\item
	$U=\min\{X,Y\}$的分布律为
		\begin{center}
			\begin{tabular}{c|cccc}
				\hline
				$U$&0&1&2&3\\
				\hline
				$p_k$&0.28&0.30&0.25& 0.17\\
				\hline
			\end{tabular}
		\end{center}
	\item
	$W=X+Y$的分布律为
	\begin{center}
		\begin{tabular}{c|ccccccccc}
			\hline
			$V$&0&1&2&3&4&5&6&7&8\\
			\hline
			$p_k$&0&0.02&0.06& 0.13&0.19&0.24&0.19&0.12&0.05\\
			\hline
		\end{tabular}
	\end{center}
	\end{enumerate}
\end{prob}