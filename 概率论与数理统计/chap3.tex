\section{多维随机变量及其分布}
\subsection{二维随机变量}
\begin{prob}
	\begin{enumerate}
	\item 	
	\renewcommand{\arraystretch}{1.7}
	\begin{tabular}{c|cc}
		\hline
		\backslashbox{$Y$}{$X$}&0&1\\
		\hline
		0& $\disp\frac{25}{36}$& $\disp\frac{5}{36}$\\
		1& $\disp\frac{5}{36}$ & $\disp\frac{1}{36}$\\
		\hline
	\end{tabular}
	\item 	
		\begin{tabular}{c|cc}
			\hline
			\backslashbox{$Y$}{$X$}&0&1\\
			\hline
			0& $\disp\frac{45}{66}$& $\disp\frac{5}{33}$\\
			1& $\disp\frac{5}{33}$& $\disp\frac{1}{66}$\\
			\hline
		\end{tabular}
	\end{enumerate}
\end{prob}
\subsection{边缘分布}
\setcounter{prob}{5}
\red{
\begin{prob}
	参见课本第2页中样本空间$S_2$的8个样本点, $X$和$Y$取值情况如下:
	\begin{center}
		\begin{tabular}{c|cccccccc}
			\hline
			样本点& $HHH$&$HHT$&$HTH$&$THH$&$HTT$&$THT$&$TTH$&$TTT$\\
			\hline
			$X$&2&2&1&1&1&1&0&0\\
			\hline
			$Y$&3&2&2&2&1&1&1&0\\
			\hline
		\end{tabular}
	\end{center}
		基于此可得$X$和$Y$的联合分布律和边缘分布律为:
		\begin{center}
			\renewcommand{\arraystretch}{1.7}
			\begin{tabular}{c|ccc|c}
				\hline
				\backslashbox{$Y$}{$X$}&0&1&2&$P\{Y=j\}$\\
				\hline
				0&$\disp\frac{1}{8}$&0&0&$\disp\frac{1}{8}$\\
				1&$\disp\frac{1}{8}$&$\disp\frac{1}{4}$&0&$\disp\frac{3}{8}$\\
				2&0&$\disp\frac{1}{4}$&$\disp\frac{1}{8}$&$\disp\frac{3}{8}$\\
				3&0&0&$\disp\frac{1}{8}$&$\disp\frac{1}{8}$\\
				\hline
				$P\{X=i\}$&$\disp\frac{1}{4}$&$\disp\frac{1}{2}$&$\disp\frac{1}{4}$&1\\
				\hline				
			\end{tabular}
		\end{center}	
\end{prob}}
\subsection{条件分布}
\setcounter{prob}{10}
\begin{prob}
\end{prob}